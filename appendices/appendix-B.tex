The result of a clustering algorithm is a set of gene clusters. In order to find out how biologically significant the cluster set is, we have used Gene Ontology \citep{GO} annotations. The GO  project is a collaborative effort to address the need for consistent descriptions of gene products in different databases. The GO collaborators are developing three structured, controlled vocabularies (ontologies) that describe gene products in terms of their associated biological processes (BP), cellular components (CC) and molecular functions (MF) in a species-independent manner. It is organized as three separate tree structured sets. The project not only writes and maintains the ontologies themselves but more importantly also makes cross-links between the ontologies and the genes and gene products in the collaborating databases. We have not used the graphical structure and inter-relationships among the terms in the ontology graph. We have used only the relationship maintained between the ontology and genes and gene products across the BP category of it as we were interested in ascertaining whether the cluster were enriched in certain biological processes.

So from this database, we extract annotations for each gene in a cluster. Then, we would like to know if any GO term is  overrepresented in the cluster compared to that happening by chance. This can be answered by \textit{p-value} from statistical hypothesis testing. The p-value is the probability of obtaining a result at least as extreme as the one that was actually observed, given that the null hypothesis is true. So, under our null hypothesis that the set of genes is randomly picked from the whole gene population we compute this p-value using a \textit{HyperGeometric} distribution as the probability that $n$ randomly chosen genes will have $k$ or more annotations of a certain type and can be written as

\[
P(X \geq k) = \sum_{i=k}^{n} \frac{\binom{K}{i} \binom{N-K}{n-i}}{\binom {N}{n}}
\]
 
Here, the total number of genes is $N$ of which $K$ are known to be of the particular annotation type that we are interested in. The cluster that we test for overrepresentation has $n$ genes.

In all our computations, we have only used GO terms that were associated with at least three genes in any cluster. Also, we have only reported those terms that had p-value less than 0.01. We excluded all clusters that were having less than 3 genes or more than 500 genes in them considering them trivial clusters. In order to correct for multiple hypothesis, we have used the False Discovery Rate with 0.05 threshold.  